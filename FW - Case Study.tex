\documentclass[a4paper,12pt]{article}
\usepackage{tikz}
\usepackage{geometry}
\usepackage{setspace}
\usepackage{tabularx}
\usepackage{xcolor}
\usepackage{titlesec}
\usepackage{enumitem}
\usepackage{graphicx}
\geometry{margin=2.5cm}
\setstretch{1.2}

\definecolor{lightblue}{RGB}{218,232,252}
\definecolor{lightgreen}{RGB}{213,232,212}
\definecolor{lightorange}{RGB}{255,230,204}
\definecolor{darkblue}{RGB}{108,142,191}

\title{\textbf{Arquitetura Simplificada do AlgoSec — Visão Técnica Detalhada}}
\author{}
\date{}

\begin{document}
\maketitle

\section*{Visão Geral}
Este documento descreve a arquitetura simplificada do \textbf{AlgoSec} em um ambiente de implantação mínima funcional.  
O objetivo é detalhar o funcionamento interno dos principais componentes (\textit{Central Manager}, \textit{Firewall Analyzer (AFA)} e \textit{FireFlow}), bem como os fluxos de comunicação, protocolos, requisitos de rede e especificações de hardware para um ambiente enxuto, porém operacional.
\usetikzlibrary{shapes, arrows, positioning}

\section*{Arquitetura}


\section*{Componentes Internos e Interações}

\subsection*{1. AlgoSec Central Manager}
\begin{itemize}[leftmargin=1.5cm]
    \item Hospeda a \textbf{GUI}, o agendador interno de tarefas, o banco de dados e o gerenciamento de usuários.
    \item Administra os módulos \textit{AFA} e \textit{FireFlow}.
    \item Gerencia integrações externas: AD, SMTP e APIs de ITSM (ex: ServiceNow).
    \item Serviços internos:
    \begin{itemize}
        \item PostgreSQL (banco de dados backend)
        \item Interface Web baseada em Tomcat (HTTPS)
        \item Daemons do AFA e FireFlow para orquestração
    \end{itemize}
\end{itemize}

\subsection*{2. AlgoSec Firewall Analyzer (AFA)}
\begin{itemize}[leftmargin=1.5cm]
    \item Realiza:
    \begin{itemize}
        \item Análise de configurações (ACLs, NAT, objetos)
        \item Identificação de regras não utilizadas
        \item Análise de risco com base em base de vulnerabilidades
        \item Simulação de conectividade (\textit{“Por que está bloqueado?”})
    \end{itemize}
    \item Compatível com mais de 100 fabricantes (Check Point, Cisco ASA, FortiGate, etc.)
    \item Comunicação:
    \begin{itemize}
        \item SSH (TCP 22) para dispositivos CLI
        \item HTTPS/REST API (TCP 443) para firewalls modernos
        \item Coleta de configurações (\textit{pull}) sob demanda ou agendada
    \end{itemize}
\end{itemize}

\subsection*{3.AlgoSec FireFlow}
\begin{itemize}[leftmargin=1.5cm]
    \item Módulo de \textbf{automação de mudanças e ticketing}.
    \item Aceita solicitações de mudança via GUI ou API.
    \item Executa automaticamente:
    \begin{itemize}
        \item Análise de risco da solicitação de regra
        \item Roteamento de aprovação com base em políticas
        \item Simulação pré-aplicação via AFA
    \end{itemize}
    \item Fluxo resumido:
    \[
    \text{Usuário} \rightarrow \text{FireFlow} \rightarrow \text{AFA} \rightarrow \text{FireFlow} \rightarrow \text{Aprovação/Implementação}
    \]
\end{itemize}

\section*{Requisitos de Segurança e Acesso}
\begin{itemize}[leftmargin=1.5cm]
    \item Acesso do AlgoSec aos Firewalls deve ser \textbf{iniciado pelo AFA}.
    \item Autenticação via:
    \begin{itemize}
        \item SSH com chave pública (recomendado)
        \item Tokens de API (para Palo Alto, Fortinet, etc.)
    \end{itemize}
    \item Integração opcional com Active Directory (LDAP/LDAPS) para RBAC.
    \item SMTP para envio de notificações.
    \item IP estático e registro DNS para o appliance AlgoSec.
    \item Operação \textbf{fora de banda} (não interfere no tráfego).
\end{itemize}

\section*{Requisitos Mínimos de Infraestrutura}

\begin{center}
\renewcommand{\arraystretch}{1.3}
\begin{tabularx}{0.95\textwidth}{|l|c|c|c|X|}
\hline
\textbf{Componente} & \textbf{vCPU} & \textbf{RAM} & \textbf{Disco} & \textbf{Observações} \\
\hline
Central Manager VM & 8 & 32 GB & 500 GB & Pode hospedar AFA + FireFlow combinados \\
\hline
Acesso de Rede & - & - & - & SSH (TCP 22), HTTPS (443), AD (389/636) \\
\hline
Plataforma OS & - & - & - & Appliance endurecido (OVA fornecido pela AlgoSec) \\
\hline
\end{tabularx}
\end{center}
\section*{ Etapas Mínimas de Configuração (Minimal Setup Steps)}
\begin{enumerate}[leftmargin=1.5cm]
    \item \textbf{Implantação do Appliance:} implantar a OVA ou ISO no ambiente VMware, KVM ou Hyper-V.
    \item \textbf{Configuração de Rede:} definir IP estático, DNS e NTP.
    \item \textbf{Adição de Dispositivos:} registrar IPs e credenciais dos firewalls.
    \item \textbf{Importação Inicial de Políticas:} executar o primeiro parsing de configuração e validar.
    \item \textbf{Criação de Solicitação de Regra de Teste:} usar o FireFlow para simular e analisar uma mudança.
    \item \textbf{Agendar Análises Periódicas:} configurar auditorias e otimizações recorrentes.
\end{enumerate}

\section*{Evolução da Arquitetura}
\begin{itemize}[leftmargin=1.5cm]
    \item Adicionar o módulo \textbf{BusinessFlow} para visualizações centradas em aplicações.
    \item Separar o \textbf{Analyzer} e o \textbf{FireFlow} em VMs distintas para melhor desempenho.
    \item Implementar \textbf{alta disponibilidade (HA)} e clusters de failover.
    \item Integrar com plataformas de \textbf{ITSM} (ServiceNow) e \textbf{SIEM} (Splunk).
\end{itemize}

\section*{Benefícios da Arquitetura Mínima}
Mesmo em sua forma simplificada, essa arquitetura entrega:
\begin{itemize}[leftmargin=1.5cm]
    \item \textbf{Visibilidade total} sobre regras e riscos.
    \item \textbf{Automação} de mudanças de política de segurança.
    \item \textbf{Relatórios centralizados} de conformidade e auditoria.
\end{itemize}
\section*{ Considerações Finais}
Esta topologia representa o modelo mínimo de implantação para um ambiente \textbf{funcional, seguro e enxuto}.  
Ela fornece base para justificar investimentos, estimar recursos e validar uma prova de conceito realista da suíte \textbf{AlgoSec}.

\end{document}
